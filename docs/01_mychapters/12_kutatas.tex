\chapter{Irodalmi áttekintés személyfelismerő módszerekről}
\label{sec:kutatas}

\section{Ember és robot interakciója}
%forrás https://ieeexplore.ieee.org/abstract/document/5975165
A gépek fejlődésével elértük azt a szintet, ahol már lehetőség nyílik az ember és robot interakciójára. Ahogy a gépek, robotok egyre több terén megjelennek az életünknek elkerülhetetlen lett, hogy tényezőként kezeljük az emberekkel való interakció képességét. Érdekes filozófiai kérdés, hogy mi is a cél és mi is az eszköz, amivel elérjük a célt. A cél, hogy emberekkel egy térben szimultán, kooperatívan működő, mozgó és feladatokat végrehajtó robotokat tervezzünk és gyártsunk, és ezen cél elérésére szolgál, hogy egy gép vagy robot képes legyen észlelni vagy felismerni egy embert? Vagy célunk az, hogy felismerjék az embert az autonóm működésű szociális robotok, mert funkciójukból adódóan elkerülhetetlen, hogy emberek interakcióba lépjenek velük és ennek nélkülözhetetlen eleme az emberek észlelése? A robotika sokoldalú felhasználásból kifolyólag mindkettő kérdés szerintem igennel válaszolható meg \cite{artc01}.

Egy automatizált gyártósoron, programozott feladatokat végrehajtó robot mellett számtalan előnnyel jár, ha egy időben egy ember is tevékenykedhet. Ezért szükséges olyan szenzorokkal felruházni ezeket a robotokat, gépeket, melyek érzékelik ember jelentlétét, ez által garantálva a mellette dolgozók biztonságát.

A gyártóknak folyamatosan javítaniuk kell a fenntarthatóságot, a termelési hatékonyságot és a minőséget a termék teljes életciklusa során, hogy biztosítsák versenyelőnyüket. Az ipari automatizálás képes a nagy hatékonyság és az ismételhetőség fenntartására a tömegtermelésben. Az együttműködő vagy kollaboratív robotokat (cobotok-at) egyre gyakrabban alkalmazzák az iparágakban az automatizálás érdekében. Kihívást jelent a cobotok fejlesztése és programozása az ipari kollaboratív feladatokra. Két különböző, nehezen megvalósítható tényezője van a fejlesztésnek: szükséges, hogy a robotok intuitívan programozhatóak legyenek, hogy a cobotok által elvégzett műveleteket a kezelő dinamikusan módosíthassa, a második tényező, hogy a cobotok-nak „tudatosan” kell műveleteiket végrehajtaniuk tekintettel a körülöttük dolgozó emberekre \cite{artc020}.

Ebben az elgondolásban a felhasználó lépett be a robot „életterébe”, ezért szükséges kooperatívvá fejleszteni a gépet. Erre szükség van egy másik jelentős piaci alkalmazás, az önvezető gépjárművek esetén is, melyek használata igen fényes jövőnek örvend személy- és áruszállításban. Ezen a területen is az emberek biztonságának garantálása a cél, melynek eszköze, hogy a konstrukciót erre képesnek tervezzük \cite{artc02}.

%forrás: https://geekflare.com/best-personal-robots/
Ebben a bekezdésben egy másik megközelítésről írok: egy gépet vagy robotot szeretnénk emberek életterébe telepíteni, mely képes bizonyos feladatok ellátására. Segítő robotok (assistive robots) feladatai lehetnek monoton munkák: ételkiszállítás, ételkészítés, takarítás elvégzése, vagy adaptív megoldásokat igénylő feladatok: tanítás, szórakoztatás vagy idősgondozás\cite{artc021}. Előnye a robotok integrációjának mindennapjainkba, hogy képesek elvégezni előre meghatározott feladataikat az emberi monotonitástűrés határain túl, magasabb precizitással. Egy robot nem fog elfáradni vagy lesz szüksége pihenőre. Képes lehet tanulni és új metódusokkal előállni. Munka végzését nem befolyásolják környezeti tényezők. Számos pozitívummal jár a robotok bevonása személyes életterünkbe, a felsorolt produktivitást növelő funkciók mellett másodlagos, ám sok kutatás témájaként megjelenő etológiai, szociális interakciók is ide tartoznak \cite{artc02}\cite{artc03}.

\section{Etológiai megközelítés}
%forrás: http://www.matud.iif.hu/2010/02/06.htm
Ahogy az előző fejezetben leírtam, a robotok egyre összetettebbek, és egyre több olyan funkció betöltését várjuk tőlük, melyben az emberi környezetben kell helytállniuk. Az emberek közösségi tereibe belépő robotok más kategóriába tartoznak, mint a gyárakban működő társaik.  Az ipari robotokkal ellentétben ezen úgynevezett szociális robotok esetében a fejlesztők törekednek arra, hogy kialakítsák azokat a kommunikációs és szociális képességeket, melyek megkönnyítik az emberek számára a robotokkal való interakciót, együttműködést. Szociális interakcióba lehet lépni, akár érzelmi kötődést kialakítani. Érdekes megközelítés, ha a háziasított állatok irányából tekintünk a robotok felé. A legelső háziasított állat a kutya volt \cite{artc040}, az első kutyák feltehetően nem feladatok ellátására lettek bevonva az emberek életközösségeibe \cite{MIKLOSI2013287}. A bevonás után jöttek rá az emberek, hogy különféle feladatok ellátására is képesek. Az szociális robotok megjelenésével húzható párhuzam. Az első emberek közé integrált robotok sem képesek még nagyobb feladatok ellátására, csupán szórakoztatás és a technológia felhasználásának bemutatására alkalmasak. Miklósi Ádám etológus írásában kifejtette a kutya szociális képességeinek kialakulását: „A háziasítás sikerességéhez feltehetően hozzájárult, hogy a kutya olyan képességeket szerzett, amelyek az egyedfejlődés során kibontakozva segítik az emberi közösségekbe való beilleszkedést.” \cite{artc04} A robotok nem képesek ilyen funckiók önálló megszerzésére, ezeket mind emberek fejlesztik. A kutyáknál eredmény volt, a robotoknál pedig cél, az hogy integráljuk őket feladataink elvégzésének megkönnyítésére. Az állatoknál az evolúció ruházta fel őket képességgel, a robotoknál viszont megvan a lehetőségünk, hogy mi ruházzuk fel őket a beilleszkedéshez szükséges szociális képességekkel \cite{artc04}.

\section{Személyfelismerés vagy észlelés}
Szakdolgozatomban emberek felismerésével foglalkozom. Az angol cím pontosabban kifejezi mi a célom: „Development of a person recognition algorithm for a mobile social robot”.  Az angol nyelv, szakirodalom különbséget tesz „detection” és „recognition” között. Az első fordítása „észlelés”, míg az utóbbié „felismerés”. Magyarul köznyelvben a két szót meglehetősen gyakran szinonímaként használjuk, ezért az elején szeretném leszögezni a különbséget. Ha egy algoritmus képes észlelni egy embert akkor arról beszélünk, hogy képes elkülöníteni a háttértől vagy többi élettelen vagy élő nem embert ábrázoló, formáló tényezőtől. A felismerés mondhatni egyel magasabb szintet képvisel bonyolultság terén. Ember felismerésére képes algoritmustól, robottól elvárás, hogy felfedezze  és megfelelően beazonosítsa az embert, egy-egy embert a másiktól elkülönítse, tehát meg tudja mondani két ember között a különbséget.


\section{Emberek észlelése}
%forrás: https://www.cyberlink.com/faceme/insights/articles/228/how-to-use-facial-recognition
Az emberek felismerésének két fontos lépése van. Az első, hogy észleljük az embert, a második, hogy megkülönböztessük a többi embertől, ellássuk egy csak rá jellemző értékkel, mondjuk névvel. Emberek észlelésével sok tudományterület foglalkozik. Rengeteg társadalmi, ipari, szociális, politikai vonatkozása van emberek észlelésének, felismerésének, jelen dolgozatban szeretnék példákat hozni ezen alkalmazásokra \cite{artc05}.

%forrás: https://rhino-partners.com/blog/computer-vision-people-detection-face-recognition/
A valós idejű emberi észlelés egyre nagyobb trenddé válik az adatkutatók körében és számos iparágban, az intelligens városoktól a kiskereskedelmen át a megfigyelő rendszerekig. Gyalogosok megszámlálása utcán vagy egy gyalogátkelőhelyen, vizsgálata annak, hogy mennyi időt töltenek el az emberek egy adott helyen, látogatók észlelése, mind olyan lehetőségek, amikből adatot lehet kinyerni emberek észlelésével. Számos ágazat, köztük a bankszektor, a biztosítók, a gyártók és mások használják az emberek észlelését, követését a fogyasztók elégedettségének javítására és marketingre. Emberek azonosítását rengeteg tényező befolyásolja: emberek testtartása és póza, fényerő, megvilágítás, időjárási viszonyok, napszakok, más tárgyakkal való interakciójuk és a kamera relatív elhelyezkedése hozzájuk képest \cite{artc06}.

A számítógépes látás segítségével történő személyazonosításnál az algoritmusnak meg kell különböztetnie azokat a részeket a képen amelyeken embereket talál, ki kell őket emelnie. Ezt követően meg kell adnia a koordinátáit és határait a potenciálisan észlelt embernek. A számítógépes látást használó algoritmusokról általánosságban elmondható, hogy a kamera által látott képet valamilyen módon szisztematikusan osztják fel, hogy kisebb egységeket kelljen átvizsgálniuk. Ennek egy módja, hogy osztályozzák a képet, olyan értelemben, hogy elkülönítik azokat a régiókat, ahol valószínűbb egy ember előfordulása. A régiókra leszűrve alkalmaznak bizonyos küszöböket. Ez azt jelenti, hogy az azonosított objektum elér-e egy olyan valószínűségi szintet, hogy azt embernek jelezze.
Emberek észlelésére alkalmazott elgondolások: arc, mozgás jellemzői, test, mély tanulás („deep learning") \cite{artc06}.

\section{Emberek felismerése}
%forrás: https://ieeexplore.ieee.org/abstract/document/7029985
A felhasználók pontos azonosítására többféle módszer létezik, nyilvánvalóan egy embert többféle tulajdonság alapján is meg lehet különböztetni egymástól. Arc, hang, testalak, arcvonások mind olyan tulajdonságok, melyek alkalmasak két ember megkülönböztetésére. Ezen tulajdonságok közül egy vagy több hiánya nem akadályozza a felismerést. Biometrikus paraméterek alkalmazásával egy bizonyos hibahatáron belül leírható egy ember. Egyértelmű, hogy tökéletes megoldás nem létezik, a paraméterek mérésére használt rendszer, szenzorok pontatlanságából, az adatfeldolgozásból, számításokból és az emberek természetes hasonlóságából mindig adódik egyfajta bizonytalanság. Mindezek mellett akár egy paraméter vizsgálatával meghatározható egy ember személye, egy bizonyos pontosággal, vagy akár többféle különböző tulajdonság egyidejű vizsgálatának összevetésével pontosítható a meghatározás. Amit elméleti szinten meg kell határozni egy személyfelismerő rendszer tervezésénél, hogy mi az a pontosság amire szükségünk van, mik azok az embereket meghatározó tulajdonságok, amiket mérni szeretnénk, illetve mekkora problémát okoz, mennyire kell kizárni, hogy két embert összekeverjünk \cite{artc07}.

Olyan területeken, ahol megkövetelt a pontosság: biztonságtechnika vagy jogi-, bűnügyi esetek, az emberek azonosítására általánosan használt módszer a DNS, az ujjlenyomat vagy az íriszminta\cite{1597098}. Hiszen ezek a legkézenfekvőbb egyénre jellemző fizikai tényezők. Biztonságtechnikailag ezek a legmeghatározóbbak. Másolásuk nagy erőfeszítést igényel. A DNS mintáról megjegyzendő, hogy könnyen használható azonosításra, de nem biztonságos, mert könnyen megszerezhető, viszont egyértelműen meghatározza az embert. Ezen módszerek elemzése nem része a dolgozatnak, csak összehasonlítás, illetve az elvek magyarázata mellett vannak tárgyalva. E módszerek előnye a pontosság, de hátrányuk, hogy kifejezetten specifikus hardver igényük van és a felismerés távolsága is limitált. Egy ujjlenyomat vagy íriszminta felvételére meglehetősen közel kell térben elhelyezkedni a robothoz\cite{artc07}.

\subsection{Felismerésre használt technológiák}
A szakdolgozatom feladatának célja egy autonóm robot, az ELTE Etológia Tanszék Biscee nevű robotjának képessé tétele emberek felismerésére és azonosítására. Ezen robotra tervezett feladatom, hogy fel tudjon ismerni embereket a környezetében. Egy felszolgáló és kutatási funkciókra szánt robot, tehát az emberekkel való interakciójának szabályai megköveteltek. Az alábbi kérdésekben tudom megfogalmazni gondolatmenetemet:
\begin{itemize}
\item Milyen eszközeink vannak, milyen szenzorok állnak rendelkezésre?
\item Milyen pontossággal szeretnénk embereket észrevenni?
\item Milyen pontossággal szeretnénk embereket megkülönböztetni egymástól?
\end{itemize}

Cél, hogy a felhasznált technológia hordozható legyen, ne igényeljen túl sok erőforrást, lehetőség szerint olyan eszközzel, szenzorral legyen végrehajtható, amivel már rendelkezik a robot, bővebben \az+\refstruc{sec:sec-cam}ban. Itt vizsgáltam a kompromisszumokat, milyen lehetőség milyen előnyöket és hátrányokat biztosít.
%TODO kamerák képek és típusok cite majd a kövi fejezetből
Az arcfelismerést választottam, mert a robot már fel van szerelve RGB kamerával, nem invazív, azaz nem kell a robotnak hozzáérni az emberhez, mint egy ujjlenyomat olvasásnál, nem szükséges közel állnia és nagy felbontású képet készíteni a szemekről, mint egy íriszminta beazonosításnál. Láb, illetve testalkat vizsgálatával kisebb pontosságot lehet elérni. Arcvonások, motívumok alapján két ember könnyen megkülönböztethető. S nem mellékes, hogy robot-etológiai szempontból a robot működését közel hozzuk az emberekéhez. Mi emberek is általában az arcokról ismerjük meg egymást.


\section{Arcfelismerés}
%forrás: https://ieeexplore.ieee.org/document/9236905
A gépi arcfelismerés már több évtizede létezik, és mostanra széles körben alkalmazzák.
Egy személy arca az ember egyedi jellemzője, ami rengeteg információt tartalmaz, például bőrszín, haj, arcszőrzet, arcforma. Ezekből az információkból és az arc gesztusaiból következtetni lehet arra is milyen a vizsgált személy hangulata, vagy éppen milyen tevékenységet folytat \cite{artc08}.

%forrás: https://www.kaspersky.com/resource-center/definitions/what-is-facial-recognition
Az arcfelismerésnek számos felhasználási módja létezik, melyek közül többet már széles körben alkalmaznak. Alkalmazási területei lehetnek: gyárak és raktárak, bank, pénzügyi szolgáltatások, irodák, tömegközlekedés, egészségügy, éttermek. Használható hozzáférési jogosultság ellenőrzésére alkalmazott és vendég megkülönböztetésére irodákban, például ajtó záraknál vagy liftekben. Egészségügyi megközelítésben ellenőrizhető a maszkviselés. Előnye, hogy érintésmentes, alkalmazási területein csökkentheti az érintésssel potenciálisan terjedő betegségek átadását. Pénzügyi szolgáltatások igénybevételét is megkönnyítheti, gyorsaságával kiválthatja hosszú jelszavak begépelését \cite{artc09}.

%forrás: https://rhino-partners.com/blog/computer-vision-people-detection-face-recognition/
\section{Eszközök arcfelismeréshez}
Ebben a fejezetben ismeretetem a legnépszerűbb arcfelismerést segítő könyvtárakat, programozói, fejlesztői eszközöket.
\subsection{OpenCV}
Az OpenCV egy platformokon átívelő könyvtár, amely valós idejű számítógépes látást használó alkalmazások készítéséhez használható. Elsősorban képfeldolgozásra, videofelvételre és -elemzésre összpontosít, olyan funkciókkal, mint az arcfelismerés és a tárgyfelismerés \cite{artc06}.

\subsection{Boof CV}
A BoofCV egy valós idejű számítógépes látáshoz írt nyílt forráskódú könyvtár. Képességei: alacsony szintű képfeldolgozás, a kamerakalibrálás, a detektálás/követés, a mozgásból származó struktúra, és az arcfelismerés. A BoofCV elérhető mind tudományos, mind kereskedelmi felhasználásra \cite{artc06}.

\subsection{TensorFlow}
A TensorFlow egy gépi tanuláshoz használható szoftverkönyvtár, amely ingyenes és nyílt forráskódú. Számos tevékenységre használható, de a mély neurális hálózatok képzésére és következtetésére összpontosít. A Google fejlesztése \cite{artc06}.


%https://medium.com/@ageitgey/machine-learning-is-fun-part-4-modern-face-recognition-with-deep-learning-c3cffc121d78